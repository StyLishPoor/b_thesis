
% \usepackage{otf} is required to use \CID{*****}

{\small
1 年間,研究活動の基礎から丁寧にご指導くださった\\
\\
\hspace{1cm}慶應義塾大学理工学部 金子 晋丈 先生\\
\hspace{1cm}慶應義塾大学理工学部 寺岡 文男 先生\\
\\
に深く感謝申し上げます.\\

また,同じ crypto チームの\\
\\
\hspace{1cm}慶應義塾大学大学院理工学研究科 前期博士課程 尾崎 耀一 氏\\
\hspace{1cm}慶應義塾大学大学院理工学研究科 前期博士課程 田中 覚士 氏\\
\hspace{1cm}慶應義塾大学理工学部 小林 うらら 氏\\
\\
に深く感謝申し上げます.みなさんの明るい雰囲気作りのおかげで,特に進捗がなくても元気にミーティングへ参加することができました.\\

研究に限らず,様々なアドバイスをしてくださった金子研究室の先輩方\\
\\
\hspace{1cm}慶應義塾大学大学院理工学研究科 後期博士課程 佐野 岳史 氏\\
\hspace{1cm}慶應義塾大学大学院理工学研究科 前期博士課程 岡 亮 氏\\
\hspace{1cm}慶應義塾大学大学院理工学研究科 前期博士課程 山下 剛志 氏\\
\\
に深く感謝申し上げます.オンライン活動しか許されないという特殊な状況の中,不自由のない研究活動を行えたのは
みなさんのおかげです.\\

残念なことに直接会話をする機会が少なかったですが,圧倒的な存在感でオンライン活動をリードしてくれた寺岡研究室の先輩方\\
\\
\hspace{1cm}慶應義塾大学大学院理工学研究科 後期博士課程 近藤 賢郎 氏\\
\hspace{1cm}慶應義塾大学大学院理工学研究科 後期博士課程 森 康祐 氏\\
\hspace{1cm}慶應義塾大学大学院理工学研究科 後期博士課程 渡\CID{14237} 大記 氏\\
\hspace{1cm}慶應義塾大学大学院理工学研究科 後期博士課程 田中 康之 氏\\
\hspace{1cm}慶應義塾大学大学院理工学研究科 前期博士課程 佐藤 友範 氏\\
\hspace{1cm}慶應義塾大学大学院理工学研究科 前期博士課程 攝待 大輔 氏\\
\hspace{1cm}慶應義塾大学大学院理工学研究科 前期博士課程 高村 壮 氏\\
\hspace{1cm}慶應義塾大学大学院理工学研究科 前期博士課程 永山 裕人 氏\\
\hspace{1cm}慶應義塾大学大学院理工学研究科 前期博士課程 和久井 拓 氏\\
\hspace{1cm}慶應義塾大学大学院理工学研究科 前期博士課程 田部 悠介 氏\\
\hspace{1cm}慶應義塾大学大学院理工学研究科 前期博士課程 嶋田 恵大 氏\\
\hspace{1cm}慶應義塾大学大学院理工学研究科 前期博士課程 長井 悠毅 氏\\
\hspace{1cm}慶應義塾大学大学院理工学研究科 前期博士課程 安森 涼 氏\\
\\
に深く感謝申し上げます.叶うことなら,みなさんと色々お喋りをしたかったです.\\

最後に,研究室の同期である\\
\\
\hspace{1cm}慶應義塾大学理工学部 大友 優 氏\\
\hspace{1cm}慶應義塾大学理工学部 中野 修平 氏\\
\hspace{1cm}慶應義塾大学理工学部 稲垣 勇祐 氏\\
\hspace{1cm}慶應義塾大学理工学部 口井 敢太 氏\\
\hspace{1cm}慶應義塾大学理工学部 藤田 滉冬 氏\\
\\
に深く感謝申し上げます.みなさんのおかげで辛くも楽しい研究室生活を送ることができました.\\
\\
最後に,研究室生活を送るにあたって支えてくださった全ての方々に深く感謝申し上げます.\\


{\small
\begin{flushright}
令和 3 年 1 月 30 日
\end{flushright}
}