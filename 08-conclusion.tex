\section{まとめ}
本研究では,空間・時間コスト削減のためにランダムウォーク(RW) によるグラフ取得をしながらノード ID を再配置する手法として,
ID の連続性を意識した Sequential と 演算時のアクセス局所性を意識した DBG-EE の 2 手法を提案した.

Sequential は RW によるグラフ取得の途中で出会ったノード順に連続した昇順の ID を再配置することで ID の連続性を保証する.
また,取得途中では RW で移動した 1 エッジの構造のみ保持することで空間コストを削減している.
また,RW による移動の仕方によって隣接ノードへ連続した ID の再配置が可能となり,近接構造によるアクセス局性が部分的に考慮されている.

DBG-EE は既存の再配置手法で次数情報のみを必要とする DBG をもとに,RW によるグラフ取得の途中でノードを次数でグループ分けし,グループ毎に ID の再配置を行う.
DBG-EE は,取得初期の段階で各グループサイズの推定を行い,グループ毎に再配置で使用する ID の範囲を決定する.
また,再配置が終了した時点で取得したグラフ構造を破棄することで空間コストを削減している.
さらに,グラフ取得と ID 再配置を並列して実行することでグラフ取得を待つだけの時間を無くし,時間コストを削減している.

そして,実世界グラフのデータセットを用いて Sequential と DBG-EE の効果をグラフ取得の完了を待たないことによる空間・時間コストの変化及びグラフ演算時間の減少率から明らかにした.
まず,DBG と比べて ID 再配置を実行するために必要なメモリ使用量は Sequential,DBG-EE ともに減少し,Sequential で最大 41.9 \% 減少した.
次に,グラフ取得開始から演算終了までの合計時間では RW 100 万回,300 万回と両方の場合で DBG-EE が DBG を下回った.
しかし,Sequential では RW 300 万回の場合で DBG より 1.1 \% 合計時間が増加した.
さらに,再配置を実行しない場合と比べて PR 演算に要する時間の減少率は DBG-EE の方が大きく,最大で 46.8 \% 減少した.
一方で PPR 演算に要する時間の減少率は Sequential の方が大きく,最大で 40.6 \% 減少した.

\section{今後の展望}
Sequential,DBG-EE ともに空間コストの大きな削減は達成したが,グラフ取得から演算終了までを合計した時間コストの大きな削減は達成できていない.
時間コストに着目したとき,RW によるグラフ取得にかかる時間を一定の制限時間とみなし,この制限時間内に最も演算時間が減少する再配置を実行する手法が望ましい.
本研究では,前処理コストの小さい DBG に着目したが,制限時間までの時間コストが許容される場合,
Gorder のような前処理コストが大きいが演算時間が大きく減少する手法に着目することが可能だと考える.

さらに,本研究では ID の連続性と 演算時のアクセス局所性にそれぞれ着目した手法を提案したが
既存手法のように ID の連続性とアクセス局所性の双方を考慮した ID 再配置が実行できれば,DBG-EE,Sequential の性能を上回ることが可能だと考える.
そこで,\ref{algo_time} 節で述べたように,対象とするアルゴリズム毎にどのようなアクセス局所性を考慮すべきか明らかにすることで,
Sequential のように ID の連続性を保証しながら,対象とするアルゴリズムに応じたアクセス局所性を最大限考慮した ID の再配置が行えると考える.