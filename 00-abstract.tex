\jabst{\large
グラフ演算においてノード ID を適切に再配置しなければ,キャッシュミスが増加し,演算速度が低下することが知られている.
分散管理されたグラフをランダムウォーク (RW) により取得する状況で既存の ID 再配置手法を適用するには,グラフの取得完了を待つ必要があり非常にコストがかかる.
また,既存の ID 再配置手法はアクセス局所性のみ考慮しているが,グラフを取得しながら再配置を行う場合は ID の連続性も考慮する必要がある.
本研究では,連続性とアクセス局所性をそれぞれ意識した Sequential と DBG Early Estimation (DBG-EE) の 2 手法を提案する.

Sequential では RW によるグラフ取得の途中で出会ったノード順に昇順の連続した ID を再配置する.
また,RW で 1 エッジ移動する度に再配置を行うことで,空間コストを大幅に削減している.
また,RW の軌跡によっては隣接ノードへ連続した ID の再配置が可能となり,部分的にアクセス局所性が考慮されている.

DBG-EE では RW 一定回数毎にノードを次数でグループ分けし,グループ毎に ID を再配置する.
なお,RW 途中で次数分布が収束する性質に着目し,既存手法の Degree Based Grouping (DBG) におけるグループ定義を踏襲している.
また,グラフ取得の途中で再配置を実行するために取得初期の段階で各グループサイズを推定し,予め各グループが使用する ID の範囲を決定する.
さらに,グラフ取得と再配置を並列して行うことで空間・時間コストを削減している.

実世界のグラフデータを使用し,提案手法の効果及びグラフ取得の完了を待たないことによるコスト削減率を評価した.
ID 再配置に必要なメモリ使用量は DBG と比べて Sequential で最大 41.9 \% ,DBG-EE で最大 35.4 \% 減少し,空間コストの削減が確認できた.
グラフ取得から PR 演算終了までの合計時間は DBG-EE が DBG を下回り,最大で 10.3 \% 減少し,時間コストの削減が確認できた.
}
\jabstfoot{}