\section{背景}
近年,コンテンツをノード,コンテンツ間の関係性をエッジとみなしたグラフ表現・解析を行うことへの注目が高まっている.
Social Network Service の友人関係や,World Wide Web の参照関係などがグラフとして表現される代表例であるが,年々これらの
グラフサイズは巨大化している.
\cite{ching2015one}では,2015 年時点で Facebook グラフのエッジ数は 1 兆を上回ると報告されており, 
この先も加速度的にグラフサイズは巨大化していくと予想される.
また,現在は単一主体によるグラフの集中管理が主流であり,コンテンツ間の関係性を定義するのはグラフの管理者である.
このような管理形態では,様々な主体が自由にコンテンツ間の関係性を見出し,その価値を流通させることは困難である.
そこで,巨大化し続けるグラフに対してスケーラビリティを確保しつつ,様々な主体が自由にコンテンツ間の関係性を定義可能なグラフ管理形態
として,自律分散グラフ管理を考える.

自律分散グラフ管理環境では,様々な主体が部分的にグラフを管理し,部分グラフの重ね合わせとして全体グラフが構成される.
各部分グラフの管理者は,管理下のコンテンツに対して自由に関係性を定義し,エッジとして保持する.
また,全体グラフを集中管理する必要はないので,グラフサイズに対するスケーラビリティも確保されている.
自律分散グラフ管理を適用したシステムの例として Catalogue \cite{catalogue}などが存在する.

自律分散グラフ管理環境において PageRank \cite{page1999pagerank} や Personalized PageRank \cite{page1999pagerank} などのグラフ演算を実行する場合,
全体グラフを演算対象とするのではなく,着目ノードから Random Walk (RW) を実行することで収集した周辺グラフを演算対象とする. 
グラフ演算において,ノード ID の配置方法はキャッシュミス発生数へ強く影響を与えるため,ノード ID を適切に再配置する手法として 
Graph Reordering が存在する.RW で収集した